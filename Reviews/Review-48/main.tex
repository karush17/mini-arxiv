
\documentclass[11pt,letterpaper]{article}
\usepackage{fullpage}
\usepackage[top=2cm, bottom=4.5cm, left=2.5cm, right=2.5cm]{geometry}
\usepackage{amsmath,amsthm,amsfonts,amssymb,amscd}
\usepackage{lastpage}
\usepackage{enumerate}
\usepackage{fancyhdr}
\usepackage{mathrsfs}
\usepackage{xcolor}
\usepackage{graphicx}
\usepackage{listings}
\usepackage{hyperref}
\usepackage{caption}
\usepackage{subcaption}
\usepackage{float}
\hypersetup{%
  colorlinks=true,
  linkcolor=blue,
  linkbordercolor={0 0 1}
}
\renewcommand\lstlistingname{Algorithm}
\renewcommand\lstlistlistingname{Algorithms}
\def\lstlistingautorefname{Alg.}

\lstdefinestyle{Python}{
    language        = Python,
    frame           = lines, 
    basicstyle      = \footnotesize,
    keywordstyle    = \color{blue},
    stringstyle     = \color{green},
    commentstyle    = \color{red}\ttfamily
}

\setlength{\parindent}{0.0in}
\setlength{\parskip}{0.05in}



\pagestyle{fancyplain}
\headheight 35pt
                 % ENTER REVIEW NUMBER HERE %
\chead{\textbf{\large Review-48}}
           % ################################### %

\lfoot{}
\cfoot{}
\rfoot{\small\thepage}
\headsep 1.5em

\begin{document}


                 % ENTER PAPER TITLE HERE %
\begin{center}
  \large{Density Estimation using Real NVP
  }
\end{center}
           % ################################### %

Unsupervised learning is hindered by the design of tractable models which must demonstrate scalability to high-dimensional tasks. The work introduces a class of real-valued non-volume preserving (NVP) models which provision tractable and exact inference, sampling and likelihood computation in the case of latent variable manipulations. NVP models utilize a set of stably invertible and learnable transformations which are found crucial for modeling natural images.

Modeling high-dimensional data requires a tractable and expressive approach. Real NVP models combine 6 novel insights from the previous work of components estimation. (1) The work employs bijective invertible transformations which are stable and flexible from the computational perspective. These transformations are incorporated as affine coupling layers wherein a tractable function is build by stacking a sequence of bijections. (2) The key insight behind this formulation is that the Jacobian determinant of transformation can be computed efficiently as the product of its diagonal terms. This results in equivalent sampling and inference costs. (3) NVP convolutions make use of partitioning to exploit the local structure across images. These correspond to convolutional masks utilized as spatial checkerboard patterns and channel-wise masking. (4) Contrary to previous works, coupling layers are combined in an alternating pattern such that the components which remain unchanged during previous transformation are updated in the subsequent one. This allows all components of forward transformations to be altered. (5) The autoregressive structure utilizes a multi-scale architecture wherein an image patch of shape $2 \times 2 \times c$ is reshaped into subsquares of shape $1 \times 1 \times 4c$ using a squeeze operation. Additionally, computational and memory costs of tensor propagation through coupling layers are reduced by factoring out half of the dimensions in a recursive fashion. (6) Lastly, the work introduces an improved training signal for deep residual networks with the running average of recent minibatches tracked in batch normalization. Application of novel batch normalization on the whole coupling layer results in robust training when training with smaller minibatch sizes. 

Real NVP models present improved performance on a range of challenging datasets when compared to prior density models. The framework is found to be competitive in the number of bits/dimensions when compared to generative methods, while not presenting significant improvements over PixelRNN. The sample quality presented by the NVP distribution is found to be consistent and locally rich in spatially challenging scenarios. For instance, while the VAE samples present occassional blurring, samples generated by NVP are sharp and globally coherent on CelebA benchmark. A similar trend is observed on ImageNet wherein the model presents a suitable representation of background and lighting interactions. However, the NVP framework does not throw light on the efficacy of each of its components, i.e.- it remains unclear as to which novel technique introduced in the model leads to spatially rich structures. Moreover, the work does not throw light on the motivation and reason behind reshaping subsquare patches during multi-scaling.

Real NVP models present two new directions for future work. Firstly, the framework lays the foundation for density estimation in light of high-dimensional data. In this regard, the work can be further extended to temporal and spatio-temporal distributions having multiple modalities. Secondly, improvements introduced in the work may be utilized to improve prior generative techniques such as VAEs and autoregressive models in light of their latent distributions. 



\end{document}
